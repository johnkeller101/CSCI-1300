\documentclass{article}

\usepackage{fancyhdr}
\usepackage{extramarks}
\usepackage{amsmath}
\usepackage{amsthm}
\usepackage{amsfonts}
\usepackage{tikz}
\usepackage[plain]{algorithm}
\usepackage{algpseudocode}
\usepackage{tikz,pgfplots,multicol,lstautogobble}

\usetikzlibrary{automata,positioning}

%
% Basic Document Settings
%

\topmargin=-0.45in
\evensidemargin=0in
\oddsidemargin=0in
\textwidth=6.5in
\textheight=9.0in
\headsep=0.25in

\linespread{1.1}

\pagestyle{fancy}
\lhead{\hmwkAuthorName}
\chead{\hmwkClass\ (\hmwkClassInstructor\ \hmwkClassTime)}
\rhead{\hmwkTitle}
\lfoot{\lastxmark}
\cfoot{\thepage}

\renewcommand\headrulewidth{0.4pt}
\renewcommand\footrulewidth{0.4pt}

\setlength\parindent{0pt}

\setcounter{secnumdepth}{0}
\newcounter{partCounter}
\newcounter{homeworkProblemCounter}
\setcounter{homeworkProblemCounter}{1}
\nobreak\extramarks{Problem \arabic{homeworkProblemCounter}}{}\nobreak{}

%
% Homework Problem Environment
%
% This environment takes an optional argument. When given, it will adjust the
% problem counter. This is useful for when the problems given for your
% assignment aren't sequential. See the last 3 problems of this template for an
% example.
%
\newenvironment{homeworkProblem}[1][-1]{
    \ifnum#1>0
        \setcounter{homeworkProblemCounter}{#1}
    \fi
    \section{Problem \arabic{homeworkProblemCounter}}
    \setcounter{partCounter}{1}
    \enterProblemHeader{homeworkProblemCounter}
}{
    \exitProblemHeader{homeworkProblemCounter}
}

%
% Homework Details
%   - Title
%   - Due date
%   - Class
%   - Section/Time
%   - Instructor
%   - Author
%

\newcommand{\hmwkTitle}{Assignment \#3}
\newcommand{\hmwkDueDate}{January 27, 2017}
\newcommand{\hmwkClass}{CSCI 1300}
\newcommand{\hmwkClassTime}{Section 100}
\newcommand{\hmwkClassInstructor}{Professor David Knox}
\newcommand{\hmwkAuthorName}{\textbf{John Keller}}

%
% Title Page
%

\title{
    \vspace{2in}
    \textmd{\textbf{\hmwkClass:\ \hmwkTitle}}\\
    \normalsize\vspace{0.1in}\small{Due\ on\ \hmwkDueDate\ at 12:30pm}\\
    \vspace{0.1in}\large{\textit{\hmwkClassInstructor\ \hmwkClassTime}}
    \vspace{3in}
}

\author{\hmwkAuthorName}
\date{}

\renewcommand{\part}[1]{\textbf{\large Part \Alph{partCounter}}\stepcounter{partCounter}\\}

%
% Various Helper Commands
%

% Useful for algorithms
\newcommand{\alg}[1]{\textsc{\bfseries \footnotesize #1}}

% For derivatives
\newcommand{\deriv}[1]{\frac{\mathrm{d}}{\mathrm{d}x} (#1)}

% For partial derivatives
\newcommand{\pderiv}[2]{\frac{\partial}{\partial #1} (#2)}

% Integral dx
\newcommand{\dx}{\mathrm{d}x}

% Alias for the Solution section header
\newcommand{\solution}{\textbf{\large Solution}}

% Probability commands: Expectation, Variance, Covariance, Bias
\newcommand{\E}{\mathrm{E}}
\newcommand{\Var}{\mathrm{Var}}
\newcommand{\Cov}{\mathrm{Cov}}
\newcommand{\Bias}{\mathrm{Bias}}

\begin{document}

\lstset{basicstyle=\fontsize{9}{10}\ttfamily,
  mathescape=true,
  escapeinside=||,
  autogobble,
  tabsize=4,
  breaklines=true,
  commentstyle=\color{gray},
  morecomment=[l]{//},
  postbreak=\raisebox{0ex}[0ex][0ex]{\ensuremath{\color{black}\hookrightarrow\space}}}

\lstset{language=C++,
                keywordstyle=\color{blue}\ttfamily,
                stringstyle=\color{red}\ttfamily,
                commentstyle=\color{gray}\ttfamily,
                morecomment=[l][\color{magenta}]{\#}
}

\begin{lstlisting}
//
//  Name            John Keller
//  Recitation TA   Jason Zietz
//  Assignment #    3
//  Problem #       1
//

#include <iostream>

using namespace std;

void madLibs(void);

int main() {
    // Declare the variables
    bool inSession = true;
    string play = "";
    
    // Loop the function so when the user is done playing, it returns them to the beginning
    while (inSession) {
        // Retrieve user input
        cout << "Do you want to play a game? (y) or (n)";
        cin >> play;
        if (play == "y") {
            // User said they wanted to play, so let's run the madLibs() function
            madLibs();
        } else if (play == "n") {
            cout << "good bye" << endl;
            inSession = false; // This boolean is what kicks the user out of the loop, ending the program
        }
    }
}

void madLibs(void) {
    // Declare our story string, including the placeholders
    string story = "In the book War of the <PLURAL NOUN>, the main character is an anonymous <OCCUPATION> who records the arrival of the <ANIMAL>s in <PLACE> -- Needless to say, havoc reigns as the <ANIMAL>s continue to <VERB> everything in sight, until they are killed by the common <SINGULAR NOUN>.";

    // Declare the titles and items to replace in story (using an array)
    string parts[6][2] = {
        {"a plural noun","<PLURAL NOUN>"},
        {"a singular noun","<SINGULAR NOUN>"},
        {"an occupation","<OCCUPATION>"},
        {"an animal name","<ANIMAL>"},
        {"a place","<PLACE>"},
        {"a verb","<VERB>"}};
    
    // Go through the arrays, retrieving a user input and replacing the placeholder string
    for (auto & part : parts) {
        // Declare our variables
        string value = "";
        string title = part[0];
        string placeholder = part[1];
        
        // Retrieve the user input
        cout << "Enter " << title << ": ";
        cin >> value;
        
        // Let's determine if the placeholder exists within the story, if so, replace it with the user's value
        // And let's put it in a loop, so we can catch repeating placeholders
        while (story.find(placeholder)!=string::npos) {
            story.replace(story.find(placeholder),placeholder.length(),value);
            // Note: .replace(startPosition, length, newString)
            //       .find(string) returns the start index of the placeholder string
            //       .length() does the obvious and returns the length
        }
    }
    
    // Print the final string
    cout << story << endl;
    
    return;
}
\end{lstlisting}

\hline

\begin{lstlisting}
//
//  Name            John Keller
//  Recitation TA   Jason Zietz
//  Assignment #    3
//  Problem #       2
//

#include <iostream>

using namespace std;

void printEnergy(double A, double r, double H, double PR);
double energyCalculator(double A, double r, double H, double PR);
int calculateNumberHousesSupported(double array_avg,double household_avg);

int main() {
    // Declare variables
    double A = 0;
    double H = 0;
    double average = 0;
    // Predefined variables:
    double r = 0.10;
    double PR = 0.75;
    
    // Retrieve the inputs for A and H
    cout << "A: ";
    cin >> A;
    cout << "H: ";
    cin >> H;
    
    // Part A
    double calculated_energy = energyCalculator(A, r, H, PR);
    cout << "The average annual solar energy production is " << calculated_energy << "kWh." << endl;
    
    // Part C
    cout << "Enter household average: ";
    cin >> average;
    cout << calculateNumberHousesSupported(calculated_energy,average) << " houses can be supported." << endl;
    
    // Part B
    while (r < 0.36) {
        printEnergy(A, r, H, PR);
        r = r + 0.05; // Add 0.05 to the previous r value
    }
}

void printEnergy(double A, double r, double H, double PR) {
    // Prints a human-readable version of the calculated energy data
    cout << "The average annual solar energy for an efficiency of " <<  r << " is " <<  energyCalculator(A, r, H, PR)  << " kWh." << endl;
}

double energyCalculator(double A, double r, double H, double PR){
    /* Calculates energy data using the equation E = A * r * H * PR
     *  A is area of the solar array in meter square
     *  r is the solar panel efficiency in percentage
     *  H is the average solar radiation in kWh/m^2/year PR is the performance ratio (coefficient of loss)
     *      - usually between 0.5 and 0.9, with a default of 0.75
     */
    return A * r * H * PR;
}

int calculateNumberHousesSupported(double array_avg, double household_avg) {
    // Perform calculations and return values
    int number_of_houses = array_avg / household_avg;
    return number_of_houses;
}

\end{lstlisting}

\end{document}
